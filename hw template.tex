% Preface: LaTeX template for homework assignments
% Author: Eurekaimer@NKU
% License: CC BY-NC-SA 4.0
% 该模板适用于撰写算法基础等课程的作业报告,包含伪代码和代码框的支持,NKU统院学生可直接使用

\documentclass[12pt, a4paper, oneside]{ctexart}

\usepackage[UTF8]{ctex}

% --- 核心宏包 ---
\usepackage{amsmath,amssymb}    % 数学公式和符号
\usepackage{algorithm}         % 伪代码环境
\usepackage{algpseudocode}     % 伪代码命令集
\usepackage{graphicx}           % 插入图片
\usepackage{float}              % 控制浮动体位置
\usepackage{enumitem}           % 增强列表功能
\usepackage{amsthm, bm, color, framed, hyperref, mathrsfs}

% --- 画框和行号宏包 ---
\usepackage{tcolorbox}
\tcbuselibrary{skins,breakable}

% --- 页面和字体设置 ---
\usepackage{geometry}
\geometry{left=1.5cm, right=1.5cm, top=1.5cm, bottom=1.5cm}
%% 控制页边距
\setlength{\parindent}{2em}     % 段落首行缩进
\setlength{\parskip}{0.5em}     % 段落间距

%% 导入lipsum宏包以生成示例文本
\usepackage{lipsum}             % 生成示例文本

% --- 问题环境 ---
\definecolor{shadecolor}{RGB}{241, 241, 255}

\newenvironment{problem}[1]{%
    \begin{shaded}\par\noindent\textbf{#1. }% 直接使用传入的参数作为标题内容
}{%
    \end{shaded}\par
}


% --- 页眉设置 ---
\usepackage{fancyhdr}
\pagestyle{fancy}
\fancyhead{}
\fancyhead[L]{姓名:XXX\quad 学号:YYYYYYYY}
\fancyhead[C]{算法基础}
\fancyhead[R]{第一次作业}

\begin{document}


\newpage

\begin{problem}{1}
    \href{https://leetcode.cn/problems/3sum/description/}{Leetcode 15 三数之和}
\end{problem}

\noindent\textbf{\large{Leetcode通过截图}}\\
\begin{figure}[H]  % [H]强制占位符在当前位置,避免浮动
    \centering  % 占位符居中对齐
    % 生成宽为页面宽度、高5cm的占位符,文字提示“图片占位符:示例图”
    \includegraphics[width=\textwidth, height=5cm, 
                     keepaspectratio=false]{example-image-a}
    \caption{这是图片的实际标题}  % 真实图片标题
    \label{fig:placeholder}  % 标签,后续可交叉引用(如“见图\ref{fig:placeholder}”)
\end{figure}

\noindent\textbf{\large{算法描述}}\\

\lipsum[3][1]  % 生成一段示例文本

\vspace{10pt}

\noindent\textbf{\large{伪代码}}\\

\begin{algorithm}[H]
    \caption{求和算法}
    \begin{algorithmic}[1]
        \Require 一个整数列表 $L$
        \Ensure 列表中所有元素的和
        \State total $\gets 0$
        \For{每个元素 $e$ 在列表 $L$ 中}
            \State total $\gets$ total $+ e$
        \EndFor
        \State \Return total
    \end{algorithmic}
\end{algorithm}

\noindent\textbf{\large{复杂度分析}}\\

\lipsum[3][2]  % 生成一段示例文本

\end{document}